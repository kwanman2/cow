\documentclass[12pt,preprint]{aastex}                         
%\documentclass[manuscript]{aastex}                           
%\documentstyle[aasms4]{article}                              
%\documentstyle[12pt,aaspp4]{article}                         
\begin{document}                                              

\def\bE{\bar{E}}
\def\bR{\bar{R}}
\def\bu{\bar{u}}

\title{Handling the Radiation Source Term Numerically}

%\author{}
\section{Notation and Statement of the Problem}

Let's start with some notation. In the lab frame, let the gas and
radiation stress-energy tensors after advection and geometrical source
terms have been applied be given by $T'^\mu_\nu$ and $R'^\mu_\nu$,
i.e., with primes on the symbols. Correspondingly, let the gas density
and internal energy at this stage of the calculation be $\rho'$, $u'$,
the gas four velocity in the lab frame be $u'^\mu_g$, the radiation
energy density be $E'_r$ and the radiation frame four velocity in the
lab frame be $u'^\mu_r$. For simplicity, let us ignore the magnetic
field.

We wish to apply the radiation source term $G_\nu$ and solve for the
final values of the various quantities. Let us describe this final
state without primes: $T^\mu_\nu$, $R^\mu_\nu$, $\rho$, $u$,
$u^\mu_g$, $E_r$, $u^\mu_r$. 

Since the continuity equation does not involve radiation, we have one
invariant quantity:
\begin{equation}
\rho' u'^0_g = \rho u^0_g = {\rm invariant}.
\label{rho}
\end{equation}
For the remaining eight quantities --- $u$, $u^i_g$, $E_r$, $u^i_r$
--- we have the following eight equations:
\begin{equation}
T^0_\nu-T'^0_\nu = G_\nu \Delta t = -R^0_\nu+R'^0_\nu,
\label{labT}
\end{equation}
where $\Delta t$ is the time step in the lab frame.  The above version
of the equations corresponds to an implicit computation, where the
radiation source term $G_\nu$ is computed based on the unprimed final
solution. In the easier explicit version, we would use $G'_\nu$.

The source term is defined in the gas rest frame. For the implicit
computation, we go to an orthonormal frame in which the gas is at
rest.  In this frame, which we identify with a ``widehat'', we have
\begin{eqnarray}
\widehat{G}^0 &=& \kappa_{\rm abs}(\widehat{E}-4\pi B), \label{G0} \\
\widehat{G}^i &=& \chi_{\rm tot}\widehat{F}^i, \label{Gi}
\end{eqnarray}
where $\widehat{E}$ is the radiation energy density in the gas frame,
and $\widehat{F}^i$ is the radiation flux vector in the same
frame. The four-vector $G_\nu$ in eq (\ref{labT}) is calculated by
transforming $\widehat{G}_\nu$ from the orthonormal gas frame to the
lab frame.

\section{Gas Frame vs Lab Frame}

Consider the following set of equations in the gas frame,
\begin{equation}
\widehat{T}^0_\nu-\widehat{T}'^0_\nu = \widehat{G}_\nu \Delta \tau = 
-\widehat{R}^0_\nu+\widehat{R}'^0_\nu,
\label{gasT}
\end{equation}
where $\Delta \tau$ corresponds to a small interval of the gas proper
time. In the gas frame, we know that the gas four velocity takes the
simple form $\widehat{u}^0_g=1$, $\widehat{u}^i_g=0$. Thus, we can
rewrite equations (\ref{gasT}) more generally as
\begin{equation}
\widehat{T}^\mu_\nu-\widehat{T}'^\mu_\nu = \widehat{G}_\nu
\widehat{u}^\mu_g\Delta \tau =
-\widehat{R}^\mu_\nu+\widehat{R}'^\mu_\nu.
\label{gasT2}
\end{equation}
This equation is in covariant form, so it is equally valid in the lab
frame:
\begin{equation}
T^\mu_\nu-T'^\mu_\nu = G_\nu u^\mu_g\Delta \tau =
-R^\mu_\nu+R'^\mu_\nu.
\label{labT2}
\end{equation}
Setting $\mu=0$ in these equations and comparing with eqs
(\ref{labT}), we see that they are identical provided
\begin{equation}
\Delta t = u^0_g \Delta \tau,
\label{tau}
\end{equation}
i.e., just the standard Lorentz transformation of time between frames.

The point of the above exercise is simply to demonstrate that
equations (\ref{labT}) and equations (\ref{gasT}) are completely
equivalent. We could solve all eight equations in the lab frame, or
all eight equations in the gas frame, or solve the $\nu=0$ pieces of
(\ref{gasT}) in the gas frame and the $\nu=i$ pieces of (\ref{labT})
in the lab frame, or any other combination of the equations.

\section{Towards a More Stable Numerical Algorithm}

Why is the previous analysis important? The source term
$\widehat{G}_\nu$ has the nasty property that its time and space
components can differ enormously in magnitude. Because of the very
steep dependence of $\kappa_{\rm abs}$ on the gas temperature $T_g$
($\equiv p/\rho = (\Gamma-1)u/\rho$), it is not unusual for
$\kappa_{\rm abs}$ to be eight orders of magnitude smaller than
$\chi_{\rm tot}$. This is no big deal in the fluid frame because the
two opacities appear in different equations. The energy equation
becomes even nicer if we follow the usual replacement trick
\begin{equation}
\widehat{T}^0_0 \to \widehat{T}^0_0+\rho \widehat{u}^0_g, \qquad
\widehat{T}'^0_0 \to \widehat{T}'^0_0+\rho' \widehat{u}'^0_g.
\end{equation}
Thus, the two energy equations in the gas frame are very well-behaved
and should provide the necessary information to solve for $u$ and
$E_r$, while the six momentum equations will have the information to
solve for the gas and radiation momentum densities. The latter could
be done either in the lab frame or the fluid frame, but it is safest
to solve the two energy equations in the fluid frame.

What is wrong with solving all the equations in the lab frame? Because
$\widehat{G}_\nu$ is defined in the moving gas frame, by the time we
transform to the lab frame, all the $\kappa_{\rm abs}$ and $\chi_{\rm
  tot}$ terms will get mixed together. If the gas is moving reasonably
quickly, say $\beta_g \sim 0.1$, and $\chi_{\rm tot}/\kappa_{\rm abs}
\gg 1$, say $10^8$, there will be a huge hit in precision. It is much
better to avoid this hit by simply solving the energy equation in the
gas frame.

Normally, even with this loss of precision, the Newton-Raphson method
would be okay. It may limp a bit, but it would not produce a crazy
solution. In the present case, we have a second problem. Because
$\kappa_{\rm abs}$ has a weird temperature dependence, or equivalently
$u$-dependence, when the value of $u$ goes a little astray during the
iteration, there is a chance for things to diverge rapidly. In RN's
opinion, this is a particularly serious problem when one works in the
lab frame, where $\chi_{\rm tot}$ mucks up the energy equation.

Here then is the recommendation:
\begin{enumerate}
\item
When applying the radiation source term, either solve all eight
equations in the fluid frame (eqs \ref{gasT} with $\Delta \tau$ given
by eq \ref{tau}), or use the two energy equations in the fluid frame
($\nu=0$ in eqs \ref{gasT}) and the six momentum equations in the lab
frame ($\nu=i$ in eqs \ref{labT}).

\item
As a further safeguard, during the iterations, do not let the gas
temperature $T_g$ and the radiation temperature $T_r \equiv
(\widehat{E}_r/a_{\rm rad})^{1/4}$ to cross each other. That is, make
sure at each step of the Newton-Raphson that $(T_g-T_r)$ has the same
sign as $(T'_g-T'_r)$. This may not be essential, but it will help a
lot to tame the wild behavior of $\kappa_{\rm abs}$.

\end{enumerate}

The above changes are likely to solve (again in RN's opinion) a large
fraction of the current problems in HARMRAD. Even if the first
recommendation does not meet with the approval of team members, there
is nothing lost by introducing the second recommendation into the
current code. It will certainly cause no harm, and it might already
(even without following the first recommendation) solve a lot of
problems.

A simpler but slightly inaccurate version of this approach, based on
entropy, was described in a previous note.

\section{Numerical Implementation}

In terms of numerical implementation, note that the primed tensors,
$T'^\mu_\nu$ and $R'^\mu_\nu$, are already available. The eight
unknowns are $u$, $u^i_g$, $E_r$, $u^i_r$.  Given a set of values for
these unknowns, we can calculate $\rho$, $T^\mu_\nu$, $R^\mu_\nu$.

Let us suppose we plan to work with the two energy equations in the
fluid frame, viz.,
\begin{equation}
\widehat{T}^0_0-\widehat{T}'^0_0 = -\kappa_{\rm abs}(\widehat{E}-4\pi B)
\Delta\tau= -\widehat{R}^0_0+\widehat{R}'^0_0.
\label{gasT00}
\end{equation}
The term $\widehat{T}^0_0$ is very straightforward,
\begin{equation}
\widehat{T}^0_0 + \rho \widehat{u}^0_g = u + b^2/2,
\end{equation}
where we have included the contribution of the magnetic energy
density, the various other $T$s and $R$s are similarly easily
calculated, e.g.,
\begin{equation}
\widehat{R}^0_0 = R^\mu_\nu u^\nu_g (u_g)_\mu, ~~{\rm etc.},
\end{equation}
while $\widehat{E}=-\widehat{R}^0_0$. Thus it is possible to compute
all the terms in the energy equations (\ref{gasT00}) without requiring
tetrads or orthonormal frames.

While everything that has been written so far is in terms of eight
unknowns and eight equations, actually there are effectively only four
unknowns. This is because, once we write down $T^\mu_\nu$ in terms of
the four gas-primitives $u$, $u^i_g$, we can immediately calculate the
radiation tensor $R^\mu_\nu$:
\begin{equation}
R^\mu_\nu = R'^\mu_\nu + T'^\mu_\nu - T^\mu_\nu,
\end{equation}
and invert to obtain the radiation primitives. The converse is also
true.  Nevertheless, solving even the resulting four non-linear
equations for the four unknowns is a major problem. In fact, when we
include magnetic fields, there are additional unknowns and certain
analytical simplifications that help in the pure hydro problem are
lost. 

A dumb, brute-force approach is to compute the local Jacobian of the
non-linear equations via numerical derivatives and to use the
Newton-Raphson method.  Jon has developed more sophisticated schemes
which reduce the nine-dimensional magnetized gas problem to five or
even four dimensions. Moreover, by computing the Jacobian
analytically, he expects the accuracy to be greatly improved. How well
these ideas will translate to the fluid-frame version of the energy
equations discussed here is unclear.  One should probably try the
basic brute-force method first before doing anything fancy.

\section{Explicit vs Implicit Solution}

It would be helpful to know when we should apply the full power of the
implicit method and when it is sufficient to simply carry out an
explicit time step. The following simplified analysis provides an
answer.

Let us consider an explicit calculation. We assume that we have
completed the advection and other non-radiative updates and have
calculated the tensors $T'^\mu_\nu$ and $R'^\mu_\nu$ and the conserved
density $\rho' u'^0_g$. Since in an explicit calculation everything is
referenced to this base state, it is not necessary to retain primes,
so we will drop the primes hereafter. Then, in the explicit approach,
the updates to the two tensors are given by
\begin{equation}
\Delta T^0_\nu = G_\nu \Delta t = -\Delta R^0_\nu.
\end{equation}
Equally well, we can write this in the fluid frame of the base state
as
\begin{equation}
\Delta \widehat{T}^0_\nu = \widehat{G}_\nu \Delta \tau = -\Delta
\widehat{R}^0_\nu.
\end{equation}

In the fluid frame, the gas has density $\rho$ (recall we decided to
drop primes), internal energy $u$, pressure $p=(\Gamma-1)u$. By
definition, the gas is not moving. The radiation, of course, will in
general be moving and will have an energy density $\widehat{E}$ and
radiation flux $\widehat{F}^i$.  Let us orient our axes so that the
radiative flux is parallel to the 1-axis. As a result of the
interaction between the gas and the radiation, the gas will be pushed
in the 1-direction and will acquire a velocity $\Delta\beta_g$. Its
internal energy, or temperature, will also change because of
absorption or emission of radiation. 

In the spirit of linearization, which is basic to the Newton-Raphson
method, $\Delta\beta_g$ can be assumed to be small and higher order
terms like $(\Delta\beta_g)^2$ can be ignored. Assuming further that
$\rho$ remains unchanged (I think this is okay but it bears thinking),
we thus have very simple expressions for the components of $\Delta
\widehat{T}^0_\nu$:
\begin{eqnarray}
\Delta \widehat{T}^0_0 = \widehat{G}_0\Delta\tau &\to& -\Delta u =
-\kappa_{\rm abs} (\widehat{E}-4\pi B)\Delta \tau, \\ \Delta
\widehat{T}^0_1 = \widehat{G}_1 \Delta\tau &\to&
(\rho+u+p)\Delta\beta_g = \chi_{\rm tot} \widehat{F}_1\Delta \tau.
\end{eqnarray}
The solution for the shifts is
\begin{eqnarray}
\Delta u &=& \kappa_{\rm abs}\Delta\tau (\widehat{E}-4\pi B),
\\ \Delta \beta_g &=& \chi_{\rm tot}\Delta\tau
\left(\frac{\widehat{E}}{ \rho+u+p}\right) \beta_r,
\end{eqnarray}
where we have assumed that the radiation frame moves slowly with
respect to the gas frame and hence written the velocity $\beta_r$ 
of this frame as
\begin{equation}
\beta_r = \widehat{F}_1 / \widehat{E}.
\end{equation} 

The two radiation equations are equally simple:
\begin{eqnarray}
\Delta \widehat{R}^0_0 = -\widehat{G}_0\Delta\tau &\to& \Delta
\widehat{E} = -\kappa_{\rm abs} (\widehat{E}-4\pi B)\Delta \tau,
\\ \Delta \widehat{R}^0_1 = -\widehat{G}_1 \Delta\tau &\to& 
\Delta \widehat{F}_1 \equiv \widehat{E} \Delta\beta_r = -\chi_{\rm tot}
\widehat{F}_1 \Delta \tau.
\end{eqnarray}
The quantity $\Delta\beta_r$ in the second equation is defined as
\begin{equation}
\Delta\beta_r \equiv \Delta\widehat{F}_1/\widehat{E}.
\end{equation}
It is a measure of how much the radiation frame velocity is changed as
a result of the explicit update to the solution. Solving the two
radiation equations gives
\begin{eqnarray}
\Delta \widehat{E} &=& -\kappa_{\rm abs}\Delta\tau (\widehat{E}-4\pi
B), \\ \Delta \beta_r &=& -\chi_{\rm tot}\Delta\tau \beta_r.
\end{eqnarray}
We thus have the complete solution for the updates under the explicit
scheme.  Now we want to know whether this solution is safe and when it
must be replaced by the more elaborate implicit solution. We use a
physically motivated criterion to answer this.

The interaction between gas and radiation is basically a relaxation
process. Consider first the energy equations. If the radiation energy
density $\widehat{E}$ exceeds the blackbody energy density $4\pi B$,
then the gas gets heated up and the radiation energy density
decreases. The converse is similarly true. Thus, the two quantities
$\widehat{E}$ and $4\pi B$ are pushed towards each other, as verified
by looking at the signs of the solutions for $\Delta u$ and
$\Delta\widehat{E}$. Physically, because we are dealing with a
relaxation process, $\widehat{E}$ and $4\pi B$ cannot cross each
other. The explicit method, however, being a local linearized
representation of the problem, knows nothing about this global
constraint, so it will sometimes produce large shifts which cause the
two energy densities to cross. We want to avoid this.  Therefore, a
simple criterion for the validity of the explicit solution is that it
should not cause crossing. This requirement is quantified as follows.

Recall that the gas temperature is given by $T_g
=(\Gamma-1)u/\rho$. Thus, the shift in $4\pi B$ caused by the explicit
step is
\begin{equation}
\Delta(4\pi B) = \Delta(a_{\rm rad}T_g^4) =  \frac{16\pi B}{u} \Delta u.
\end{equation}
The net shift in $(\widehat{E}-4\pi B)$ is then estimated to be
\begin{equation}
|\Delta(\widehat{E}-4\pi B)| = \kappa_{\rm abs}\Delta\tau\left(1+
\frac{16\pi B}{u}\right)(\widehat{E}-4\pi B).
\end{equation}
For stability of the explicit scheme, we want the left hand side to be
no more than some fraction $\zeta$ ($<1$) of $(\widehat{E}-4\pi
B)$. Thus, we obtain the following criterion:
\begin{equation}
{\rm Criterion~1:}\qquad
\kappa_{\rm abs}\Delta\tau\left(1+ \frac{16\pi B}{u}\right) < \zeta.
\label{crit1}
\end{equation}

The momentum equation is again a relaxation process.  If the radiation
is moving towards the positive 1-direction ($\beta_r>0$), then the gas
acquires a positive velocity ($\Delta\beta_g>0$), while the radiation
velocity decreases ($\Delta\beta_r <0$). The two frames thus approach
each other, but they cannot cross. However, as in the previous case,
the explicit scheme does not know about this condition.  Therefore, a
second criterion is that $\Delta\beta_g-\Delta\beta_r$ should be less
than $\zeta$ times $\beta_r$. This gives the following second
criterion:
\begin{equation}
{\rm Criterion~2:}\qquad
\chi_{\rm tot}\Delta\tau\left(1+ \frac{\widehat{E}}{\rho+u+p}\right) < \zeta.
\label{crit2}
\end{equation}
When both criteria (\ref{crit1}) and (\ref{crit2}) are satisfied, we
expect the explicit scheme to be well-behaved. However, if either or
both are violated, we should switch to the implicit scheme.

What value of $\zeta$ should we choose? Any value less than unity is
probably fine. However, we made several approximations in our
analysis, so it may be good to be conservative.  A value of
$\zeta=0.5$ is probably sufficient, while $\zeta=0.1$ should be quite
safe.

An interesting sidelight of the above analysis is that, apart from the
obvious factors $\kappa_{\rm abs}\Delta\tau$ and $\chi_{\rm
  tot}\Delta\tau$, two other factors turn out to be important: $16\pi
B/u$ and $\widehat{E}/(\rho+u+p)$. This gives a clue to the behavior
of the equations. Consider the energy equation first. When $16\pi B
\gg u$, the explicit shift acts primarily on the gas (through $u$) and
has only a small effect on the radiation ($\widehat{E}$). In this
limit, it is preferable to use $u$ as our primary primitive variable
and to solve for $\widehat{E}$ in terms of it, rather than the other
way round.  On the other hand, if $16\pi B \ll u$, then it is better
to use $\widehat{E}$ as the primary primitive and treat $u$ as a
derived quantity.  In the case of the momentum equation, the critical
ratio is $\widehat{E}/(\rho+u+p)$. If $\widehat{E} \gg (\rho+u+p)$,
there is a large velocity shift in the gas and a negligible velocity
shift in the radiation. We should, therefore, treat $\beta_g$ as our
primary primitive. In the opposite limit, we should treat $\beta_r$ as
the primary primitive.

\end{document}
